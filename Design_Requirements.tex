% Design Requirements
\chapter{Design Requirements}

The following requirements are based on the problem statement, team brainstorming, user needfinding, and prototypes described in the design development section. Some of this requirements also come from a geometrical and structural analysis of the plane (cabin and cargo hold) that are detailed in Appendix F.
\\
\\This requirements fall into two categories: Functional and Physical. Functional requirements detail what a design should do, its actions and capabilities. Physical requirements describe the constraints on the manifested system components. 
\\
\\ We have divided the functional and physical requirements into two main parts:
\begin{list}{-}{}
  \item Requirements for the wheelchair platform that will improve wheelchair storage from the jetway to the cargo hold
  \item Requirements for the transfer mechanism that will enable wheelchair passengers to reach their seats in a more comfortable and independent way
\end{list} 

\section{Functional Requirements}

\subsection*{Wheelchair storage device}

In order to prevent wheelchairs from being damaged our team wants to design a platform that will enable baggage handlers to move the mobility device from the jetway to the cargo hold. This platform will be composed of four parts:

\begin{list}{-}{}
  \item A pallet that will support the wheelchair
  \item A moving mechanism to enable transport from jetway to cargo hold
  \item Electronics that will identify who is handling the wheelchair and detect mishandling
  \item A handle that will enable baggage handlers to interact with the platform
\end{list}

The functional requirements for each of this part are in the tables from \ref{tab:functional_requirements_platform} to \ref{tab:functional_requirements_electronics}.

\newpage

\begin{table}
\begin{tabular} {| p{4cm} | p{5cm} | p{5cm} |}
\hline
\textbf{Requirement} & \textbf{Metric} & \textbf{Rationale} \\ \hline

Support Load & Platform must be able to support a minimum of 450 lb & Platform must support the weight of the power wheelchair and the passenger \\ \hline
Lightweight & Platform weighs less than 60 lbs & Platform must not add substantial weight to the airplane's payload \\ \hline
Wheelchair Climbable & Platform is less than 1.5 inches high & The wheelchair cannot climb on top of the platform if it is too high  \\ \hline
Platform Structure & Platform must not deflect more than .5 inches when the wheelchair is on top & Platform must be reinforced in places where it is likely to deform \\ \hline
External wheels & Platform does not use wheelchair wheels for mobility & Handlers often break wheelchairs by trying to shift their gears \\ \hline
Transition from Movable to Stationary & Needs an actuator that activates moving mechanism in less than 20s & Needs to shift from moving to stationary quickly depending on the situation.  \\ \hline
Time efficient & Putting the wheelchair on top of the platform and securing it should take less than 5min & Baggage handlers and airport personnel are busy, since they also have to deal with luggage and other passengers \\ \hline
Wheelchair Attachment to Platform & Platform has straps for the wheelchair with placement conforming to transportation regulations:  36 CFR 1192.23, D2
& Wheelchair must be rigidly attached to platform using a system that wheelchair users already trust and will provide them with peace of mind\\ \hline
Fatigue & Platform must be able to withstand 5000 cycles & Platform should be replaced sparingly. \\ \hline
\end{tabular} 
\caption{Functional Requirements for the Platform}
\label{tab:functional_requirements_platform}
\end{table}

\clearpage
\newpage

\begin{table}
\begin{tabular} {| p{4cm} | p{5cm} | p{5cm} |}
\hline
\textbf{Requirement} & \textbf{Metric} & \textbf{Rationale} \\ \hline

Moving Load &  Wheels have to support at least 200 pounds & The heaviest power wheelchairs can be up to 200 lbs \\ \hline
Resistant Wheels & Wheels have to withstand and overcome friction forces found in rough terrain &	Our system will be used on the tarmac as well as on the jetway \\ \hline
Withstand stationary load &	Parts in contact with the floor must be able to support 450 lbs	 & Person will be driving the wheelchair on to the platform thus it will have to withstand the weight of power wheelchair + person (max 250lbs) \\ \hline
\end{tabular} 
\caption{Functional Requirements for the Moving Mechanism}
\label{tab:functional_requirements_moving}
\end{table}

\begin{table}
\begin{tabular} {| p{4cm} | p{5cm} | p{5cm} |}
\hline
\textbf{Requirement} & \textbf{Metric} & \textbf{Rationale} \\ \hline

Support Load & Handle must be able to move a load of at least 200 lbs without deflecting more than 1in & Handler needs to be assured that the handle can push the required load  \\ \hline
Ergonomics & Handle must comply with standards for pushing heavy items & Handler must exert minimal force to move platform along  \\ \hline
Electronics Housing & Capability to hold electronics & Handlers need an easy and quick access to the RFID antenna.  \\ \hline
\end{tabular} 
\caption{Functional Requirements for the Platform Handle}
\label{tab:functional_requirements_handle}
\end{table}

\clearpage
\newpage

\begin{table}
\begin{tabular} {| p{4cm} | p{5cm} | p{5cm} |}
\hline
\textbf{Requirement} & \textbf{Metric} & \textbf{Rationale} \\ \hline
Handler Check-in & Electronics include an RFID antenna that recognizes the RFID tag in their gloves & Handlers' accountability will increase if they know that others are aware that the wheelchair is their responsibility \\ \hline
Instantaneous Feedback & 	Feedback for successful check in is received within 10 ms & Handler must immediately know when they have successfully checked in so they can continue doing their jobs\\ \hline
Visual Feedback & Electronics includes highly visible LEDs & Visual feedback is easy to spot. Handlers are in a loud environment and are wearing thick gloves so auditory and tactile feedback are out of the question.\  \\ \hline
Detect Free Fall and Mishandling & Electronics include an accelerometer that is calibrated to detect any acceleration greater than 14 G’s that are shorter than 12.5 ms in duration & Airlines needs to be notified of any falls or hard impacts so they may appropriately prepare and notify the wheelchair user \\ \hline
Wireless Connection & Electronics have embedded wireless communication capabilities & Passengers should know who is handing their wheelchair and airlines should know who is responsible when mishandling occurs so they may take action \\ \hline
Easily Rechargeable & Battery should only need to be charged for 5-6 hours at a time & Battery should be able to get by only being charged during airport down-times (e.g. overnight) \\ \hline

\end{tabular} 
\caption{Functional Requirements for the Platform Electronics}
\label{tab:functional_requirements_electronics}
\end{table}

\clearpage
\newpage











\subsection*{Wheelchair transfer mechanism}

In order to improve the boarding experience for disabled passengers, our team decided to redesign the aisle wheelchair by making it possible for a wheelchair user to transfer himself from the wheelchair to his seat without needing to be carried by flight attendants. The system that we are designing to give wheelchair users their independence back will be composed of seven parts:

\begin{list}{-}{}
  \item A chest rest that will equip the aisle wheelchair in order to make its use more comfortable and less degrading for passengers
  \item A locomotion mechanism that would enable wheelchair users to move through the cabin
  \item A cushion that will support disabled passengers
  \item A sliding base that will enable transfer from the aisle wheelchair to the airplane seat without external assistance
  \item A wheelchair structure that will completely be redesigned to accommodate the sliding base and the chest rest
  \item A footrest that will enable disabled passengers to have support for their legs
  \item A new airplane seat that will be adapted to the sliding base
\end{list}

The functional requirements for each of this part are in the tables from \ref{tab:functional_requirements_chestrest} to \ref{tab:functional_requirements_wheelchair}.

\newpage

\clearpage
\newpage

\begin{table}
\begin{tabular} {| p{4cm} | p{5cm} | p{5cm} |}
\hline
\textbf{Requirement} & \textbf{Metric} & \textbf{Rationale} \\ \hline

Removable from Base &  There is a latching mechanism on the sliding base to attach/detach chest rest & The rows must be unobstructed to allow other passengers to exit in an emergency and the chestrest must be available for other passengers that may request the aisle wheelchair. \\ \hline
Adjustable Height & Minimum vertical variation is 6in & It has to be adjustable to people with different body types\\ \hline
User Support &   The mechanism shall keep a 150kg sandbag securely positioned during transfers & The mechanism shall provide safety and keep the user erect and attached to the structure \\ \hline
User Comfort & There shall not be contact areas with a pressure above 200 mmHg & We want to avoid bruises or sores on the user's  body \\ \hline
Body Support & Rigid structure should withstand a 330lb person leaning on it & It allows the user to feel secure while being transferred
\\ \hline
Ergonomic Height & Handles located 30in above ground & Allows assistant to easily push/pull the wheelchair \\ \hline
\end{tabular} 
\caption{Functional Requirements for the Chest Rest}
\label{tab:functional_requirements_chestrest}
\end{table}


\begin{table}
\begin{tabular} {| p{4cm} | p{5cm} | p{5cm} |}
\hline
\textbf{Requirement} & \textbf{Metric} & \textbf{Rationale} \\ \hline


Steering Radius &  Wheelchair can steer in a 15in radius & Allows for maneuvering inside the tight spaces in the plane \\ \hline
Brakes & Wheelchair does not move more than 5mm after being locked (under normal circumstances). & Allows personnel to securely lock the wheelchair to avoid accidents during transfer  \\ \hline
\end{tabular} 
\caption{Functional Requirements for the Locomotion Mechanism}
\label{tab:functional_requirements_locomotion}
\end{table}

\begin{table}
\begin{tabular} {| p{4cm} | p{5cm} | p{5cm} |}
\hline
\textbf{Requirement} & \textbf{Metric} & \textbf{Rationale} \\ \hline

Foot Support &  Has a support for passengers feet with dimensions of 15in x 6 in & To prevent the feet from being dragged. \\ \hline
Foot Securing & Has a mechanism that keeps the user's feet away from obstacles on the aisle. & To keep the user's feet secure while moving the wheelchair.  \\ \hline
\end{tabular} 
\caption{Functional Requirements for the Footrest}
\label{tab:functional_requirements_footrest}
\end{table}


\clearpage
\newpage


\begin{table}
\begin{tabular} {| p{4cm} | p{5cm} | p{5cm} |}
\hline
\textbf{Requirement} & \textbf{Metric} & \textbf{Rationale} \\ \hline

Adjustable & It should be at least 4" tall and the filling/stuffing should be adjustable (different pressure/padding)& Its common for wheelchair users to suffer injuries for seating for long periods of time. Cushion must be adjustable to user to provide them with comfort \\ \hline
Removable & There is a detachable mechanism on the cushion & Allows for proper hygienization and maintenance. \\ \hline
\end{tabular} 
\caption{Functional Requirements for the Cushion}
\label{tab:functional_requirements_cushion}
\end{table}


\begin{table}
\begin{tabular} {| p{4cm} | p{5cm} | p{5cm} |}
\hline
\textbf{Requirement} & \textbf{Metric} & \textbf{Rationale} \\ \hline
Non-Aligned Lateral Transfer & Allows a transfer with a gap of 2in and a .4in tolerance between the seat's cushion base and the wheelchair's cushion base & Makes the lateral transfer easier and faster  \\ \hline
Independent Transfer & A maximum 50N force will be required for the transfer & Allows for user to independently transfer themselves without the help of others  \\ \hline
Secure & Has a latching device that latches it on to the seat and wheelchair & Avoids undesired lateral transfers and consequently accidents. \\ \hline
\end{tabular} 
\caption{Functional Requirements for the Sliding Base}
\label{tab:functional_requirements_sliding}
\end{table}

\begin{table}
\begin{tabular} {| p{4cm} | p{5cm} | p{5cm} |}
\hline
\textbf{Requirement} & \textbf{Metric} & \textbf{Rationale} \\ \hline

Maximum Load & Must withstand a 150kg load & Wheelchair must be secure enough to be used by passengers. Due to the restriction of the size of the aisle of the airplane, people that are extremely obese were not considered on this design. \\ \hline
Chair Mobility & Chair has wheels that can move with minimal force & The chair must be able to easily move inside the aircraft. \\ \hline
\end{tabular} 
\caption{Functional Requirements for the Wheelchair Structure}
\label{tab:functional_requirements_wheelchair}
\end{table}

\clearpage
\newpage

\subsection{Functional Constraints}

\subsection*{Wheelchair storage device}

\begin{list}{-}{}
  \item Due to FAA regulations and flight requirements the platform protecting the wheelchair must withstand 3 G's.
  \item The platform cannot damage the space or items within the space during its operation.
  \item The electronics will require a power source (battery) to perform some of its functions.
  \item The battery should satisfy the DOT’s Hazardous Materials Regulations (HMR; 49 CFR parts 100-185).
  \item The entire device needs to be approved as safe for air travel.
\end{list}

\subsection*{Wheelchair transfer mechanism}

\begin{list}{-}{}
  \item Due to FAA regulations and flight requirements every element that goes inside the cabin must withstand 6 G's.
  \item The redesigned aisle wheelchair must be as safe as the standard aisle wheelchair operated today.
  \item The new aisle wheelchair has to operate in the constrained space of the airplane aisle.
  \item In case of turbulence, the wheels of the new aisle wheelchair must be locked.
\end{list}

\subsection{Functional Assumptions}

\subsection*{Wheelchair storage device}

\begin{list}{-}{}
  \item Since the straps that are used to attach the wheelchair to the platform are the same as bus straps, wheelchair users are assumed to already be familiar with them. The use of this mechanism should provide them peace of mind since they already trust it. 
  \item The platform configuration changes (rests on the wheels to be moved or rests on the platform to be stored) will be triggered by baggage handlers acting on an electric jack.
  \item Baggage handlers will be wearing their gloves with the RFID tag every time  they handle a wheelchair and use our platform.
\end{list}

\subsection*{Wheelchair transfer mechanism}

\begin{list}{-}{}
  \item All aisle seats  that will be used by wheelchair users must have retractable armrest. It must allow the mechanism to slide without any obstacles in its way.
  \item The passenger will always be accompanied by a flight attendant to guarantee the safety of the passenger and to pull/push the aisle wheelchair.
  \item Tetraplegic passengers will always be accompanied by a travel companion/assistant. Because of regulation (ANAC) and their reduced autonomy, tetraplegic users are not allowed to fly alone.
  \item The airplane seat must withstand 19 G's during takeoff and landing. This is what the FAA requires in order to make sure seats can withstand any type of accident happening during takeoff or landing.
\end{list}

\subsection{Functional Opportunities}

\subsection*{Wheelchair storage device}

\begin{list}{-}{}
  \item Our platform must be intuitive to use. A baggage handler should know how to attach a wheelchair to the platform after a 15 min training.
  \item Ideally the platform should have a braking system for the wheels to prevent it from moving too fast on a sloped surface. A braking device would allow handlers to have better control of the device.
\end{list}

\subsection*{Wheelchair transfer mechanism}

\begin{list}{-}{}
  \item Although we are designing for a paraplegic user, the transfer mechanism shall also allow a tetraplegic user to be transferred from the aisle wheelchair to his/her seat with assistance. In this case, it should be a more pleasant experience since the assistant would not need to lift the tetraplegic user like they do today.
  \item The chest rest makes frontal transfer possible. Today, this cannot be done due to the current aisle chair dimensions.
  \item It reduces the risk and liability associated with transfer because no one has to carry the disabled passenger. The chest rest is here to avoid human contact that can be unpleasant and/or source of accidents.
  \item The chest rest can facilitate paraplegic passengers using the restroom.
\end{list}


\section{Physical Requirements}

The physical requirements of the system explain the physical appearance and structures chosen for each part of the devices. 

\subsection*{Wheelchair storage device}

The physical requirements for the wheelchair storage device can be found between Tables \ref{tab:physical_requirements_electronics} and \ref{tab:physical_requirements_handle}.

\clearpage
\newpage

\begin{table}
\begin{tabular} {| p{4cm} | p{5cm} | p{5cm} |}
\hline
\textbf{Requirement} & \textbf{Metric} & \textbf{Rationale} \\ \hline

Non-intrusive RFID Tag & Tag is built into handler's current equipment and cannot be felt when not in use. We have selected to imbed the RFID tag into the gloves &
Tag should seamlessly integrate into their current equipment in order to reduce friction \\ \hline
Intuitive Check-in & There is a mark on the appropriate finger that is to be used to check-in & Wheelchair handler needs to knows exactly what finger to use to check-in \\ \hline
Thin Electronics Enclosure & Enclosure thickness cannot be greater than .125in & RFID tag needs to be max .5” away to be recognized by the antenna \\ \hline
Package Modularity & Enclosure can be opened and electronics can be removed and replaced & 
Electronics need to be removable in case battery power runs out or a component needs to be replaced  \\ \hline
Easily Located & Electronics box is stands out because it is bright blue and is located on platform handle &
Handler needs to be able to easily find the electronics box in order to check in. \\ \hline
Attachable & Enclosure is attached to the handle & Electronics must travel with the wheelchair platform at all times and be easily accessible by the handler \\ \hline
Lightweight Enclosure & Enclosure does not weigh more than .5 lbs & Every ounce of weight costs the airline money
 \\ \hline
\end{tabular} 
\caption{Physical Requirements for the Electronics}
\label{tab:physical_requirements_electronics}
\end{table}

\begin{table}
\begin{tabular} {| p{4cm} | p{5cm} | p{5cm} |}
\hline
\textbf{Requirement} & \textbf{Metric} & \textbf{Rationale} \\ \hline

Floor Safety  & Surface area in contact with the floor is greater than 24in\textsuperscript{2} &
Reduces probability of cabin floor failure  \\ \hline
Lightweight & Weighs less than 60lbs & Each extra lb costs money to the airline  \\ \hline

\end{tabular} 
\caption{Physical Requirements for the Platform}
\label{tab:physical_requirements_platform}
\end{table}


\clearpage
\newpage

\begin{table}
\begin{tabular} {| p{4cm} | p{5cm} | p{5cm} |}
\hline
\textbf{Requirement} & \textbf{Metric} & \textbf{Rationale} \\ \hline
Handle & Platform has an attached handle & Wheelchairs are often broken due to handling mistakes. By providing the handler with a handle independent of the wheelchair that they can use to move the wheelchair from the jetway to the cargo hold,  we are reducing the chance of a mishandling accident \\ \hline
Wide Handle & Platform has a 3ft wide attached handle & The wider handle allows for the handler to place his/her arms farther apart and exert more force to move the wheelchair. \\ \hline
Handle Height & Handle is at a height of 4 ft 6 in above the ground & Handle is at a height where the average person can exert the most amount of force with least amount of effort according to the Center for Occupational Health and Safety \\ \hline
\end{tabular} 
\caption{Physical Requirements for the Handle}
\label{tab:physical_requirements_handle}
\end{table}

\subsection*{Wheelchair transfer mechanism}
The physical requirements for the wheelchair transfer mechanism can be found between Tables \ref{tab:physical_requirements_chestrest} and \ref{tab:physical_requirements_sliding}.


\clearpage
\newpage


\begin{table}
\begin{tabular} {| p{4cm} | p{5cm} | p{5cm} |}
\hline
\textbf{Requirement} & \textbf{Metric} & \textbf{Rationale} \\ \hline

Inclusive Design &  Clean design that does not dehumanize passenger &  Inconspicuous design improves user experience. \\ \hline
Ergonomic & Adapted to average user size and body shapes & Incites people to use it because it looks well designed and comfortable  \\ \hline
\end{tabular} 
\caption{Physical Requirements for the Chest Rest}
\label{tab:physical_requirements_chestrest}
\end{table}

\begin{table}
\begin{tabular} {| p{4cm} | p{5cm} | p{5cm} |}
\hline
\textbf{Requirement} & \textbf{Metric} & \textbf{Rationale} \\ \hline
Retractable Handles & Handles retract at least 15cm & Allows for the lateral movement of the system and avoids handles getting stuck on the seat's backrest\\ \hline
\end{tabular} 
\caption{Physical Requirements for the Locomotion Mechanism}
\label{tab:physical_requirements_lococomotion}
\end{table}

\begin{table}
\begin{tabular} {| p{4cm} | p{5cm} | p{5cm} |}
\hline
\textbf{Requirement} & \textbf{Metric} & \textbf{Rationale} \\ \hline
User Comfort
 & Has a special cushion that protects user from hard edges on mechanism & Avoid injuries in contact with rigid surfaces and mechanism.\\ \hline
\end{tabular} 
\caption{Physical Requirements for the Cushion}
\label{tab:physical_requirements_cushion}
\end{table}


\begin{table}
\begin{tabular} {| p{4cm} | p{5cm} | p{5cm} |}
\hline
\textbf{Requirement} & \textbf{Metric} & \textbf{Rationale} \\ \hline
Attachable Chest Rest & Has a latching mechanism & Chest rest must be easily secured and removed\\ \hline
\end{tabular} 
\caption{Physical Requirements for the Sliding Base}
\label{tab:physical_requirements_sliding}
\end{table}
\clearpage
\newpage

\subsection{Physical Constraints}

\subsection*{Wheelchair storage device}

\begin{list}{-}{}
  \item The wheelchair load (more than 600 lbs for the heaviest wheelchairs) must be distributed over a minimum surface area of 8 in\textsuperscript{2} otherwise the stress experienced by the cargo hold floor will be too high and the floor may collapse. (All the details explaining how we got the 8in\textsuperscript{2} figure are shown in appendix F).
  \item For the RFID identification system, the distance between the antenna and the tag must be as small as possible. The outer plastic surface covering the RFID antenna must be less than .125''.
\end{list}

\subsection*{Wheelchair transfer mechanism}

\begin{list}{-}{}
  \item The width of the aisle (19.75'' in Embraer jets E175) limits the size of the aisle wheelchair.
  \item The new aisle wheelchair must be less than or equal to current aisle wheelchairs which weigh approximately 35 lbs.
\end{list}

\subsection{Physical Assumptions}

\subsection*{Wheelchair storage device}

\begin{list}{-}{}
  \item If a ramp is used for the wheelchair to get on the platform, it has to be removed once used otherwise it would take too much space in the cargo hold.
\end{list}

\subsection*{Wheelchair transfer mechanism}

\begin{list}{-}{}
  \item We assume that the average wheelchair passenger using our transfer mechanism will have an average weight of 180 lbs.
  \item If disabled passengers have the opportunity to use the new redesigned aisle wheelchair to go to the restroom, we assume that this restroom is accessible and must have lateral grab bars.
\end{list}

\newpage

\subsection{Physical Opportunities}

\subsection*{Wheelchair storage device}

\begin{list}{-}{}
  \item The handle that enables baggage handlers to move the platform should be ergonomic. They should be covered by an ergonomic sleeve made of material such as silicon in order to increase comfort of pushing the platform around.
\end{list}

\subsection*{Wheelchair transfer mechanism}

\begin{list}{-}{}
  \item Wheelchair transfer from the passenger's wheelchair to the new aisle wheelchair will happen once the passenger has driven his/her mobility device on top of the platform. This way, baggage handlers do not manipulate the wheelchair at all. They only touch the platform. This should provide peace of mind to our user.
  \item Wheelchair users that are overweight should feel more comfortable in our redesigned aisle wheelchair because there is more support on the sides.
\end{list}


