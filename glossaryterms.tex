  \item  \textbf{80/20:} Aluminum T-slotted profiles used for building modular structures. 

\item  \textbf{ADA:} Americans with Disabilities Act; one of America's most comprehensive pieces of civil rights legislation that prohibits discrimination against and guarantees people with disabilities have the same opportunities as everyone else to participate in the mainstream of American life.

 \item \textbf{Aisle Chair:} Common assistive device utilized to help individuals with mobility limitations to more easily board airplanes. Aisle chairs are narrower wheelchairs which highlight multiple straps to completely secure the user, and can be rolled down narrow airplane aisles to get the individual to his or her seat. 

  \item \textbf{Airport Personnel:} All parties involved in handling the passengers and cargo for a flight including but not limited to: flight attendants, luggage handlers, check-in personnel and other contracted helpers.

  \item \textbf{ANAC:} Agencia Nacional de Aviação Civil – Brazilian National Agency of Civil Aviation

  \item \textbf{ANSYS:} ANSYS Mechanical software is a comprehensive finite element analysis tool for structural analysis, including linear, nonlinear and dynamic studies. The engineering simulation product provides a complete set of elements behavior, material models and equation solvers for a wide range of mechanical design problems.

\item \textbf{Arduino:} A low cost and easy to use microcontroller for rapid prototyping of mechatronic and electrical systems

  \item  \textbf{Assistive Technology:} Assistive, adaptive, and rehabilitative devices for people with disabilities; promotes greater independence by enabling people to perform tasks that they were formerly unable to accomplish, or had great difficulty accomplishing.

  \item  \textbf{Benchmarking:} A standard by which something can be measured or judged.

\item \textbf{Boarding:} The process of entering the airplane.

  \item  \textbf{Cabin:} The section of an aircraft in which passengers travel. 

 \item  \textbf{Cargo Hold:}  The space in a ship or aircraft for storing cargo such as baggage, shipping containers, animals or mobility devices. 

  \item \textbf{Control:} The power to influence or direct either people's behavior or the course of events.

  \item \textbf{Dark Horse Prototype:} A device created during the winter quarter of ME310 that was ruled out in the fall quarter or undiscovered due to being “too risky” or “to difficult to complete”; emphasizes creative out-of-the-box thinking and exploring all of the design space for the project. 

  \item \textbf{Disability:} A physical or mental condition that limits a person's movements, senses, or activities.

\item \textbf{Disembarking:} The process of exiting the airplane.

  \item \textbf{EXPE:} The Stanford design fair that is held every year at the beginning of June. During this fair, all ME 310 teams present the work they have done throughout the year and show their final prototype.

  \item \textbf{FAA:} Federal Aviation Administration; United States national aviation authority whose mission is to provide the safest, most efficient aerospace system in the world, oversees all aspects of American civil aviation.

  \item \textbf{Independence:} Freedom from outside control or support.

\item \textbf{Jetway:} A telescoping corridor that extends from an airport terminal to an aircraft, for the boarding and disembarkation of passengers.

  \item \textbf{Limited Mobility:} Mobility impairment may be caused by a number of factors, such as disease, an accident, or a congenital disorder and may be the result from neuro-muscular or orthopedic impairments. It may include conditions such as spinal cord injury, paralysis, muscular dystrophy and cerebral palsy. It may be combined with other problems as well (i.e. brain injury, learning disability, hearing or visual impairment).

  \item \textbf{Needfinding:} Discovering opportunities by recognizing the gaps in the system or the needs.

  \item \textbf{Non-Discriminatory:} Fairness in treating people without prejudice.

  \item \textbf{Pain Points:} A level of difficulty sufficient to motivate someone to seek a solution or an alternative; a problem or difficulty.

  \item \textbf{Perspective:} A particular attitude toward or way of regarding something; a point of view.

  \item \textbf{Self-Image:} The idea one has of one's abilities, appearance, and personality.


 \item \textbf{Taxiing:} The movement of an aircraft on the ground, under its own power.

  \item \textbf{Transfer:} The act of moving a wheelchair user from one chair to another. 

